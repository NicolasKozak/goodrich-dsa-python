%! Author = lair001
%! Date = 8/10/2022

% Preamble
\documentclass[11pt]{article}

% Packages
\usepackage{amsmath}
\usepackage{indentfirst}

% Document
\begin{document}

    \section*{Prompt}\label{sec:prompt}

\textbf{\textit{Postfix notation}} is an unambiguous way of writing an arithmetic expression without parentheses.
    It is defined so that if $``(\mathit{exp_1})\mathbf{op}(\mathit{exp_2})"$ is a
normal, fully parenthesized expression whose operation is $\mathbf{op}$, the postfix
version of this is $``\mathit{pexp_1} \mathit{pexp_2} \mathbf{op}"$, where $\mathit{pexp_1}$ is the postfix version of
$\mathit{exp_1}$ and $\mathit{pexp_2}$ is the postfix version of $\mathit{exp_2}$. The postfix version of a single number or
    variable is just that number or variable. For example, the
postfix version of $``((5+2) \ast (8-3))/4"$ is $``5 \; 2 \; + \; 8 \; 3 \; - \; \ast \; 4 \; /"$. Describe
a nonrecursive way of evaluating an expression in postfix notation.

    \pagebreak

    \section*{Discussion}\label{sec:discussion}
    See the solution for Exercise 6.34.

\end{document}